\documentclass[prc,amsart,english]{revtex4}
\usepackage[T1]{fontenc}       % DC-fonts
\begin{document}
\title{Training in Advanced Low Energy Nuclear Theory}
\maketitle
\section{Introduction to the Talent Courses}
A recently established initiative, {\bf TALENT: Training in Advanced
Low Energy Nuclear Theory}, see \url{www.nucleartalent.org},  aims at
providing an advanced and comprehensive training to graduate students
and young researchers in low-energy nuclear theory. The initiative
is a multinational network between several European and Northern
American institutions and aims at developing a broad curriculum that
will provide the platform for a cutting-edge theory for understanding
nuclei and nuclear reactions. These objectives will be met by
offering series of lectures, commissioned from experienced teachers
in nuclear theory. The educational material generated under this
program will be collected in the form of WEB-based courses,
textbooks, and a variety of modern educational resources. No such
all-encompassing material is available at present; its development
will allow dispersed university groups to profit from the best
expertise available.

The advanced training network in nuclear theory will provide students (theorists and experimentalists) 
with a broad background in methods and techniques that can be easily applied to other domains of science and technology. The characteristic feature of this initiative is training in multi-scale nuclear physics. This knowledge is crucial, not only for a basic understanding of atomic nuclei, but also for further development of knowledge-oriented industry; from nanotechnology and material science to biological sciences, to high performance computing. As such, the proposed training aims at providing an inter-disciplinary education when it comes to theories and methods.

The ultimate goal of the proposal is to develop a graduate program of excellence in low-energy nuclear theory. The program will build strong connections between universities and research laboratories and institutes worldwide and provide a unique training ground for the future needs of nuclear physics.

The first course is now being planned and will run at the ECT* in Trento, Italy, from June 25 to July 13 2012. What follows below is a detailed description of this course.

\section{Course on Computational Many-body Methods for Nuclear Physics}

\subsection{Aims and Learning Outcomes}
The aim is to learn how to solve complicated  quantum many-body problems
beyond mean field approximations using advanced numerical methods. The course aims also at understanding and implementing
numerical methods and modern computational facilities.  The acquired skills will enable the participants to write their own
codes and to work and understand existing codes for solving complicated many-body problems.
This will give the participants the necessary knowledge for tackling  a broader spectrum of research problems.
The course will also focus on how to write a scientific report via a final assignment which will be graded.
\subsection{Course Content}
This is an advanced course on computational physics with an emphasis on quantum mechanical
systems with many interacting particles. The applications and the computational methods are
relevant for research problems in such diverse areas as nuclear, atomic, molecular and solid-state physics, chemistry and materials science.
A theoretical understanding of the behavior of quantum-mechanical many-body systems - that is, systems containing many interacting particles - is a considerable challenge in that no exact solution can be found; instead, reliable methods are needed for approximate but accurate simulations of such systems on modern computers. New insights and a better understanding of complicated quantum mechanical systems can only be obtained via large-scale simulations. The capability to study such systems is of high relevance for both fundamental research and industrial and technological advances.

The aim of this course is to present, through various computational projects, applications of some of the most widely used many-body methods with pertinent algorithms and high-performance computing topics such as advanced parallelization techniques and object orientation. The methods and algorithms that will be studied may vary from year to year depending on teachers and the interests of the participants, but the main focus will be on nuclear physics related methods.

\subsubsection{Detailed Course Content}
The first course on computational methods will run at the ECT* in Trento, in collaboration with the
University of Trento, starting June 25 and ending July 13 in 2012. The course will focus on Monte Carlo
methods (Variational and Diffusion Monte Carlo) and large-scale diagonalization methods (Configuration
interactions) as solvers for the many-body problem. The system will be a simplified representation of nuclei, with protons and neutrons (or spin $1/2$ fermions) described by a harmonic oscillator potential
in three dimensions and interacting via central Yukawa and Coulomb interactions.
This system contains the basic features needed to describe nuclei and is simple enough to let the students
develop programs to solve Schr\"odinger's equation for many interacting (charged) particles with the above methods.

The detailed content of the course is
\begin{enumerate} 
\item[First week]
\begin{itemize}
\item The problem per se, basic many-body physics, hamiltonians,
setup of Slater determinants and single-particle basis.
Rehearsal of basic many-body physics and examples of computation of Hamiltonian matrices (Monday first week).
\item The Monte Carlo part: Basics of stochastic methods: Central Limit Theorem, sampling by Markov Chains, random numbers, covariance,
auto-correlation functions and error estimates,  blocking for data analysis. (Tuesday first week)
\item Wave functions: General structure of a correlated many-Fermion wave function, Slater determinants
and the Feenberg expansion for correlations.
The concept of local energy and its computation: efficient calculation of  Slater determinants, its derivatives,
and the Jastrow factor. Improved Monte Carlo methods (force-biased, Langevin). (Wednesday, Thursday and Friday of first week)
\end{itemize}
\item[Second week]
\begin{itemize}
\item Parallelization: standard master-slave (using Open multi-processing (OpenMP) 
and Message-passing interface (MPI)) and
parallel I/O for blocking analysis (Monday of second week). 
\item Imaginary time propagators, Trotter-Suzuki method, analogy with the diffusion equation. Drift, diffusion
and branching. Importance sampling in DMC. Fixed node approximation (Tuesday and Wednesday second week).
Additional topics covered by exercises: Optimization methods like conjugate gradient plus correlated samplings
\item The large-scale diagonalization part:
Shell-model and no-core shell model, basic philosophies, effective interactions and
similarity transformations. (Thursday of second week).
\item Diagonalization algorithms, basics of Krylov methods with Lanczos algorithm for symmetric matrices,
Givens and Householder's algorithms  for smaller systems, typically with dimensionalities less
than $10^5$ basis states.
Convergence properties and other mathematical properties of iterative methods. (Friday of second week)
\end{itemize}
\item[Third week]
\begin{itemize}
\item Set up of a Slater determinant in $m$-scheme. Discussion of different angular momentum recoupling schemes.
Action of the Hamiltonian on the basis of Slater determinants, actual implementation of the Lanczos
algorithm (Monday and Tuesday of third week).
\item Parallelization of the Lanczos algorithms and intermediate orthogonalization using MPI and OpenMP (Wednesday of third week).
\item Computation of expectation values, energies and transition operators (Thursday of third week ).
\item Summary of course and discussion of possible assignment (Friday of third week).
\end{itemize}
\end{enumerate}
The course ends with a final assignment. The final assignment will be graded
with marks A, B, C, D, E and failed for Master students and passed/not passed for PhD students.

\subsection{Teaching}
The course will be taught as an intensive  course of duration of three weeks, with a
total time of 45 h of lectures, 45 h of exercises and  a final assignment of 2 weeks of work.
The total load will be approximately 240 hours, corresponding to  {\bf 10 ECTS} in Europe.
Organization of the day: 9-12 lectures, time for exercises with assistance (including lunch)
till 18 (3 hours of allocated exercise sessions per day).
Summary and questions and student presentations till 19.
The course will be run at the premises ECT* (Villazzano) in Trento, Italy, from
June 25 to July 13 in 2012.

\subsection{Teachers}
Morten Hjorth-Jensen (Michigan State University and University of Oslo), Calvin Johnson (San Diego State University, California), Francesco Pederiva (University of Trento) and Kevin Schmidt (Arizona State University, Phoenix, USA)
Responsible for the course: Morten Hjorth-Jensen.
\subsection{Prerequisites}
The students are expected to have operating programming skills in Fortran/C++/Python and knowledge of
quantum mechanics at an intermediate level. \\
Preparatory modules on second quantization, Wick's theorem,
representation of Hamiltonians and calculations of Hamiltonian matrix elements, independent particle models and Hartree-Fock theory will be provided prior to the course start. Students who have not studied the above topics are expected to gain this knowledge prior to attendance. \\
Additional modules for self-teaching on
Fortran and/or C++, parallelization, and basic knowledge of parallelization and
libraries like MPI, Lapack and Blas will also be provided.

\subsection{Admission}
The target group is Master of Science students, PhD students and early post-doctoral fellows.
Attendance should be both experimentalists and theorists.
Also senior staff can attend but they have to be self-supported.
The maximum number of students is 20, of which only at most 15 can
receive full local support. The lecture hall can accomodate 10
additional  self-supported participants, without the desk space.
The process of selections of the students will be managed in agreement with the University of Trento.

\section{More Material on the Talent Program}
\subsection{Other courses planned}

Course 1: Nuclear forces and their impact in nuclear structure


Course 2: Many-body methods for nuclear physics


Course 3: Few-body methods and nuclear reactions


Course 4: Density functional theory and self-consistent methods


Course 5: Theory for exploring nuclear structure experiments


Course 6: Theory for exploring nuclear reaction experiments


Course 7: Nuclear theory for astrophysics


Course 8: Theoretical approaches to describe exotic nuclei


\end{document}










